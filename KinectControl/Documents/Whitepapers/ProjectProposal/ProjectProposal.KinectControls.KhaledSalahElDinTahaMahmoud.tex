\documentclass[a4paper]{article}

\usepackage[T1]{fontenc}
\usepackage[adobe-utopia]{mathdesign}
\usepackage[protrusion=true,expansion=true]{microtype}
\usepackage{xcolor}
\usepackage{relsize}

\definecolor{darkblue}{rgb}{0,0,0.5}

\usepackage[colorlinks=true,
        urlcolor=darkblue,
        anchorcolor=darkblue,
        linkcolor=darkblue,
        citecolor=darkblue,
        pdfauthor={Khaled Osmaan},
        pdfkeywords={case study, model driven design, Kinect, assisted
        living, home devices},
        pdftitle={Kinect Control---Controling home devices with Kinect},
        pdfsubject={Bachelor topic proposal for the faculty for MET,
          the German University in Cairo GUC (http://met.guc.edu.eg/)}]{hyperref}
\usepackage{url}

\author{Khaled Osmaan}
\title{Kinect Control\\{\smaller Controling home devices with
    Kinect\\{\smaller A project proposal}}}

\begin{document}

\maketitle

\begin{abstract}
  Today's technology is getting smarter producing amazing new
  opportunities and making our lives easier. This project provides an
  easy way to control home devices using Microsoft kinect sensor and a
  home automation device by using kinect's voice recognition, Skeleton
  tracking and depth tracking. It also provides an interface the user
  can interact with to edit and show the statuses of the controlled
  home devices.

  Some might think you can just get up an shut the light or the device
  off manually, but that's not the point, people are lazy, it's all
  about removing obstacles. What if when you enter a room the lights
  turn on and when you go to bed everything turns off. You can have
  certain lighting configuration for studying, watching t.v, reading,
  etc\ldots\ saves energy, time, effort and removes obstacles.
\end{abstract}

\section*{Project data}

\begin{description}
\item[Supervisor name:] Dr.\ Georg Jung
\item[Student's name:] Khaled Salah ElDin Taha Mahmoud
\item[Student's Major:] Computer Science---MET---Engineering
\item[Student's ID:] 19-2558
\end{description}

\section*{Tasks}
\begin{itemize}
\item Install the needed sensors if any and make the connection
  between the devices and the home automation device, If no rooms
  could be used then show a visual simulation of what happens or use
  an arduino board to turn on certain output devices instead
  (examples: LEDS, Motors, LCD).
\item Implement a gesture recognizer to recognize certain body/hand
  gestures.
\item Implement voice recognition.
\item Implement a screen manager to switch screens and add them on top
  of each others to be able to implement the interface.
\item Implement an XNA application which is the user interface the
  user can interact with directly.
\item Implement how the interaction between the kinect sensor and the
  home automation device or the arduino board will be done.
\end{itemize}



\end{document}

%%% Local Variables: 
%%% mode: latex
%%% TeX-master: t
%%% End: 
